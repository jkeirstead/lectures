%% Example lecture file
%% 
%% The content of the lecture, both notes and slides, are written in 
%% this single document.

%% Header information
\lecture[1]{An introduction to widgets}{intro-widgets}
\course[WD 101]{Widgets for Beginners}
\date{9 September 2011}
\author{Prof.\ J.\ Bloggs}
\email{j.bloggs@university.ac.uk}
\institute{Big University}

%% Begins the main document
\begin{document}

%% Make the title slide
\begin{frame}
\maketitle
\end{frame}

%% Use the section commands to divide up the talk, inserting slides where appropriate. 

\section{Introduction}
Today's talk will be about widgets.  

%% A basic frame with some content
\begin{frame}{Widgets}
A widget is a thingy.
\end{frame}

%% A frame with a subtitle and columns
\begin{frame}{Widgets}{In two columns}
\begin{columns}[t]
\begin{column}{0.5\textwidth}
\begin{itemize}
\item First column bullets
\end{itemize}
\end{column}
\pause % Pauses second column display 
\begin{column}{0.5\textwidth}
Second column text \citep{Body2000}
\end{column}
\end{columns}
\end{frame}

% Some code
\begin{frame}[fragile]{Sample R code}
\begin{rcode}
x <- 1
\end{rcode}
\end{frame}

%% Optional slide for references
\section{Further reading}
If you would like to read more about widgets, please try these references.
\begin{frame}[allowframebreaks]{References}
\bibliographystyle{plainnat}
\bibliography{refs}
\end{frame}

\end{document}
