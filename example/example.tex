%% Example lecture file
%% 
%% The content of the lecture, both notes and slides, are written in 
%% this single document.

%% Header information
\lecture[1]{An introduction to widgets}{intro-widgets}
\subtitle{Optional subtitle}
\course[WD 101]{Widgets for Beginners}
\date{9 September 2011}
\author[Bloggs]{Prof.\ J.\ Bloggs}
\email{j.bloggs@university.ac.uk}
\institute{Big University}

%% Begins the main document
\begin{document}

%% Make the title slide
\begin{frame}
\maketitle
\end{frame}

%% Use the section commands to divide up the talk, inserting slides where appropriate. 

\section{Introduction}

% Any text that is not contained within a frame environment
% will be displayed only in the article version of the lecture.
Today's talk will be about widgets.  

% Some frames (slides)
%-----------
% This is a standard Beamer frame with titles
\begin{frame}{Widgets}
A widget is a thingy.
\end{frame}

% A frame with a subtitle and columns
\begin{frame}{Widgets}{In two columns}
\begin{columns}[t]
\begin{column}{0.5\textwidth}
\begin{itemize}
\item First column bullets
\end{itemize}
\end{column}
\pause % Pauses second column display 
\begin{column}{0.5\textwidth}
Second column text \citep{Body2000}
\end{column}
\end{columns}
\end{frame}

% Code display
%--------------
% More code environments can be defined above
% using the listings package.
\begin{frame}[fragile]{Sample R code}
\begin{rcode}
x <- 1
\end{rcode}
\end{frame}

% Hidden output
% --------------
% Occasionally you will want to only display
% a blank space so that students can fill in a gap during
% the lecture.  This is done with the quiz environment

\begin{frame}{A quiz}
A quick quiz: $1 + 1 = ?$
\begin{quiz}
The answer is 2
\end{quiz}
\end{frame}

% References
% ----------
%% Optional slide for references
\section{References}
If you would like to read more about widgets, please try these references.

\begin{frame}[allowframebreaks]{References}
\bibliographystyle{plainnat}
\bibliography{refs}
\end{frame}

\end{document}
